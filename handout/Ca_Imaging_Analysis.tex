
%%%%%%%%%%%%%%%%%%%%%%%%%%%%%%%%%%%%%%%%%
% Short Sectioned Assignment
% LaTeX Template
% Version 1.0 (5/5/12)
%
% This template has been downloaded from:
% http://www.LaTeXTemplates.com
%
% Original author:
% Frits Wenneker (http://www.howtotex.com)
%
% License:
% CC BY-NC-SA 3.0 (http://creativecommons.org/licenses/by-nc-sa/3.0/)
%
%%%%%%%%%%%%%%%%%%%%%%%%%%%%%%%%%%%%%%%%%

%----------------------------------------------------------------------------------------
%	PACKAGES AND OTHER DOCUMENT CONFIGURATIONS
%----------------------------------------------------------------------------------------

\documentclass[paper=a4, fontsize=11pt]{scrartcl} % A4 paper and 11pt font size

\usepackage[T1]{fontenc} % Use 8-bit encoding that has 256 glyphs
\usepackage{fourier} % Use the Adobe Utopia font for the document - comment this line to return to the LaTeX default
\usepackage[english]{babel} % English language/hyphenation
\usepackage{amsmath,amsfonts,amsthm} % Math packages
\usepackage{gensymb}
\usepackage{chemmacros}
\usepackage{sectsty} % Allows customizing section commands
\allsectionsfont{\centering \normalfont\scshape} % Make all sections centered, the default font and small caps

\usepackage{fancyhdr} % Custom headers and footers
\pagestyle{fancyplain} % Makes all pages in the document conform to the custom headers and footers
\fancyhead{} % No page header - if you want one, create it in the same way as the footers below
\fancyfoot[L]{} % Empty left footer
\fancyfoot[C]{} % Empty center footer
\fancyfoot[R]{\thepage} % Page numbering for right footer
\renewcommand{\headrulewidth}{0pt} % Remove header underlines
\renewcommand{\footrulewidth}{0pt} % Remove footer underlines
\setlength{\headheight}{13.6pt} % Customize the height of the header

\numberwithin{equation}{section} % Number equations within sections (i.e. 1.1, 1.2, 2.1, 2.2 instead of 1, 2, 3, 4)
\numberwithin{figure}{section} % Number figures within sections (i.e. 1.1, 1.2, 2.1, 2.2 instead of 1, 2, 3, 4)
\numberwithin{table}{section} % Number tables within sections (i.e. 1.1, 1.2, 2.1, 2.2 instead of 1, 2, 3, 4)

\setlength\parindent{0pt} % Removes all indentation from paragraphs - comment this line for an assignment with lots of text

%general text processing
\newcommand{\supr}[1]{\ensuremath{^{#1}}}
\newcommand{\sub}[1]{\ensuremath{_{#1}}}

%----------------------------------------------------------------------------------------
%	TITLE SECTION
%----------------------------------------------------------------------------------------

\newcommand{\horrule}[1]{\rule{\linewidth}{#1}} % Create horizontal rule command with 1 argument of height

\title{	
\normalfont \normalsize 
\textsc{Universit\"{a}t Basel, Mrsic-Flogel Lab} \\ [25pt] % Your university, school and/or department name(s)
\horrule{0.5pt} \\[0.2cm] % Thin top horizontal rule
\huge Ca\supr{++} Imaging Analysis in MATLAB
\horrule{1.5pt}\\ % Thick bottom horizontal rule
}

\author{}

\date{\normalsize\today} % Today's date or a custom date

\begin{document}

\maketitle % Print the title

In this practical you will analyse 2-photon imaging data using MATLAB. This is the same approach 
neuroscientists use to analyse data in their studies. You should complete all steps by modifying the 
MATLAB scripts you have been provided. 

In this practical you will be supplied with with raw Ca\supr{++} imaging data from mouse V1. The mouse 
was anesthetized and was stimulated drifting visual gratings of different orientations. From these data
we can obtain an orientation tuning curve for each cell in the field of view. Your task will be to to 
extract the raw fluorescence intensity trace from one cell, convert this to dF/F, and then 
determine the dF/F for each orientation in order to produce a polar plot that shows response magnitude
for different orientations. You will then be able to select different cells and quickly pass the extracted 
response timecourses through the functions you have written to get orientation tuning curves for several 
different cells. 



\subsection{Start MATLAB and download the data}
\begin{itemize}
\item Download the data needed for this practical from \textit{http://mouse.vision/ca.zip} and unpack the zip archive. 
\item Start MATLAB.
\item Use the change directory button at the top of the screen (SCREENSHOT) to navigate to the 
directory containing the files you just unzipped. 
\end{itemize}


\subsection{Extract response time-course from a cell}
\begin{itemize}
\item Load the image stack called \texttt{Calcium\_imaging\_data\_int8.tif} using the \texttt{XXXX} command (I THINK SHOULD WE SUPPLY A
READER FUNCTION FOR MULTI-PAGE TIFFS)
\item Calculate the average image using the \texttt{mean} command and assign it to a variable called \texttt{meanIM}.
\item Start image selection GUI using the \texttt{XXXXX} command and load the \texttt{meanIM} image. (WE NEED TO GET THEM TO SAVE IMAGES SOMEWHERE)
\item Look at the image. What does this projection tell you about the activity of different neurons?
\item Look at the movie. What do you see?
\item Draw an ellipse around a cell to highlight it. GENERATE MASK.
\item Calculate average image intensity from the ROI for each frame. Plot the resulting time course. (IMAGE OF WHAT IT SHOULD LOOK LIKE)
\item You will now compute the \textit{change} in fluorescence over time (dF/F)
	\begin{enumerate}
	\item Load the file  \texttt{calc\_dF\_F.m}. This gets as input the raw trace from one cell and returns as output the dF/F. You will edit
	the function to calculate and return the dF/F.
	\item dF/F = F-F\sub{0}/F\sub{0}. Fo is the median of the fluorescence (F) distribution. Calculate this in the function file using the texttt{median} command.
	\item Subtract F\sub{0} value from each fluorescence (F) value, and then divide the resulting value by F\sub{0}. 	
	\item Run your function and plot the dF/F. SAVE YOUR IMAGE
	\end{enumerate}
\end{itemize}


\subsection{Understanding the stimulus presentation paradigm}
Compute the average response timecourse over multiple repetitions of the same stimulus. In this dataset, each of the 16 stimulus orientations was presented 3 times (The drifting grating was presented for 1.5 seconds preceded by a gray screen presented for 3.5 seconds). The stimulus orientation is indicated by the number in the first column. SEE THE IMPORTTRACE FUNCTION. 

\begin{itemize}
\item Draw below how this stimulation paradigm looks like by indicating the time in seconds for baseline and for stimulus presentation for an example orientation presentation (one repetition).\vspace{3in}
\item Using the number of seconds for one repetition and the number of frames per repetition (see spreadsheet THEY CAN LOOK AT THIS IN THE VARIABLE EDITOR IF NEEDED OR WE GET THEM TO PLOT IT). Estimate the imaging frame rate. Add the number of frames on your drawing, above.
\item Average the three traces for each orientation and plot these average traces for each of the 16 stimulus orientations on the same plot. 
\item Plot a single average trace for all orientations. Where does the peak fall with respect to the stimulus start time? Refer to your diagram, above?
\end{itemize}

\subsection{Calculate responses to different stimuli}
You will now find the mean response to each of the 16 stimulus orientations. EDIT FILE XXX
\begin{itemize}
\item Average the 5 time points around the peak response (during the stimulus period) from the average timecourse of each stimulus orientation. 	
\item Plot the mean responses on a polar plot, which will reveal the orientation tuning of the cell. Why is a polar plot a better choice than a conventional x/y line or scatter plot?
\item Compute orientation selectivity index. Find the stimulus orientation that caused the biggest mean response (‘preferred’ stimulus). Find the mean response of the two stimuli that are 90 degree away from the preferred stimulus (‘orthogonal’ stimulus), and average the two values. Orientation selectivity index (OSI) is calculated as the: OSI = (preferred - orthogonal) / (preferred + orthogonal)
\end{itemize}

\subsection{Repeating for other cells}
You should now have a series of files that you can call in sequence to get a polar plot. FILE A THEN FILE B THEN FILE C.... If you select a new cell, you can pass it through these functions to quickly generate a polar plot. 

\begin{itemize}
\item Create a new function called \texttt{batch.m}. Provide as an input argument the response time course. 
\item In the function body call the functions you have made in order. Passing the correct inputs and outputs to each so that 
you can feed \texttt{batch.m} a response time course and get back a polar plot. 
\item Run \texttt{batch.m} for 5 different cells and save the resulting plots. 
\end{itemize}



\end{document}